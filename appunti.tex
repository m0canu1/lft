\documentclass{article}

\usepackage{mathtools}

\begin{document}

\section{Alfabeto, Stringhe, Linguaggi}
\subsection{Alfabeto}
Un Alfabeto è un \textbf{insieme finito} di elementi detti \textbf{simboli} o \textbf{caratteri}.\\
La \textbf{cardinalità} è il numero di simboli dell'alfabet.
\subsection{Stringhe}
La \textbf{stringa vuota} è indicata con $\epsilon$.
\subsubsection{Operazioni sulle stringhe}
\subparagraph{Concatenazione}
Il simbolo è il punto $(.)$ tra stringhe:\\
"nano.tecnologie" diventa "nanotecnologie".
\subparagraph{Riflessione}
Consiste nello scrivere una stringa al contrario, ovvero invertire l'ordine dei suoi simboli (caratteri).\\
$x^R$ denota la riflessione della stringa $x$.\\
La riflessione della concatenazione di due stringhe è la concatenazione inversa delle loro riflessioni:\\
$(xy)^R=y^Rx^R$
\subparagraph{Potenza m-esima}
La potenza della stringa $x$ è al concatenazione di se stessa $m$ volte.\\
La \textbf{potenza} ha la \textbf{precedenza} sul concatenamento:\\
$abbc^3 = abbccc$
\subsection{Linguaggi}
\subsubsection{Operazioni sui linguaggi}
\subparagraph{Concatenazione}
Il concatenamento di due linguaggi $L_1$ ed $L_2$ (notazione $L_1L_2$) è l'insieme ottenuto
concatenando in \textbf{tutti i modi possibili} le stringhe di $L_1$ con le stringhe di $L_2$.
\begin{equation*}
    L_1L_2=\{x\ |\ x=yz\ \&\ y \in L_1\ \&\ z \in L_2\}
\end{equation*}
$\{ab,\ abc\}\{ab,\ aa,\ cb\}=\{abab,abaa,abcb,abcab,abcaa,abccb\}$
\section{Automi finiti ed Espressioni regolari}
\subsection{Automi finiti}
\subsubsection{DFA - Automa Finito Deterministico}
È una quintupla:
\begin{equation*}
    A=\{Q,\Sigma,\delta,q0,F\}
\end{equation*}
\subparagraph*{$Q$} è un insieme finito di \textbf{stati}.
\subparagraph*{$\Sigma$} è un alfabeto finito (\textbf{simboli in input}).
\subparagraph*{$\delta$} è una funzione di transizione $Q\times\Sigma\rightarrow Q$
\subparagraph*{$q0$} $\in\ Q$ è lo \textbf{stato iniziale}.
\subparagraph*{$F$} è l'insieme degli \textbf{stati finali}.

La funzione di transizione $\delta$ si può estendere alle stringhe:
\begin{equation*}
    \hat{\delta}:Q\times\Sigma^*\rightarrow Q
\end{equation*}
quindi:
\begin{equation*}
    \hat{\delta}(q,\epsilon)=q
\end{equation*}
\begin{equation*}
    \hat{\delta}(q,xa)=\delta(\hat{\delta}(q,x),a)  
\end{equation*}
\end{document}