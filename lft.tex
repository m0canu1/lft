\documentclass[12pt]{extarticle}

\usepackage{mathtools}
\usepackage{tikz}
\usetikzlibrary{automata,positioning,arrows}

\begin{document}

\section{Alfabeto, Stringhe, Linguaggi}
\subsection{Alfabeto}
Un Alfabeto è un \textbf{insieme finito} di elementi detti \textbf{simboli} o \textbf{caratteri}.\\
La \textbf{cardinalità} è il numero di simboli dell'alfabet.
\subsection{Stringhe}
La \textbf{stringa vuota} è indicata con $\epsilon$.
\subsubsection{Operazioni sulle stringhe}
\subparagraph{Concatenazione}
Il simbolo è il punto $(.)$ tra stringhe:\\
"nano.tecnologie" diventa "nanotecnologie".
\subparagraph{Riflessione}
Consiste nello scrivere una stringa al contrario, ovvero invertire l'ordine dei suoi simboli (caratteri).\\
$x^R$ denota la riflessione della stringa $x$.\\
La riflessione della concatenazione di due stringhe è la concatenazione inversa delle loro riflessioni:\\
$(xy)^R=y^Rx^R$
\subparagraph{Potenza m-esima}
La potenza della stringa $x$ è al concatenazione di se stessa $m$ volte.\\
La \textbf{potenza} ha la \textbf{precedenza} sul concatenamento:\\
$abbc^3 = abbccc$
\subsection{Linguaggi}
\subsubsection{Operazioni sui linguaggi}
\subparagraph{Unione}
\begin{equation*}
    A\cup B = \{x|x\in A\ or\ x\in B\}
\end{equation*}
\subparagraph{Concatenazione}
Il concatenamento di due linguaggi $L_1$ ed $L_2$ (notazione $L_1L_2$) è l'insieme ottenuto
concatenando in \textbf{tutti i modi possibili} le stringhe di $L_1$ con le stringhe di $L_2$.
\begin{equation*}
    L_1L_2=\{x\ |\ x=yz\ \&\ y \in L_1\ \&\ z \in L_2\}
\end{equation*}
$\{ab,abc\}\{ab,aa,cb\}=\{abab,abaa,abcb,abcab,abcaa,abccb\}$
\subparagraph{Star}
\begin{equation*}
    A*=\{x_1x_2x_3\dots x_k\ |\ k\geq 0\ and\ ogni\ x_i \in A\}
\end{equation*}
\newpage
\section{Automi finiti ed Espressioni regolari}
\subsection{Automi finiti}
\subsubsection{DFA - Automa Finito Deterministico}
È una quintupla:
\begin{equation*}
    A=\{Q,\Sigma,\delta,q0,F\}
\end{equation*}
\subparagraph*{$Q$} è un insieme finito di \textbf{stati}.
\subparagraph*{$\Sigma$} è un alfabeto finito (\textbf{simboli in input}).
\subparagraph*{$\delta$} è una funzione di transizione $Q\times\Sigma\rightarrow Q$
\subparagraph*{$q0$} $\in\ Q$ è lo \textbf{stato iniziale}.
\subparagraph*{$F$} è l'insieme degli \textbf{stati finali}.

La funzione di transizione $\delta$ si può estendere alle stringhe:
\begin{equation*}
    \hat{\delta}:Q\times\Sigma^*\rightarrow Q
\end{equation*}
Definizione:
\begin{equation*}
    \hat{\delta}(q,\epsilon)=q
\end{equation*}
\begin{equation*}
    \hat{\delta}(q,xa)=\delta(\hat{\delta}(q,x),a)  
\end{equation*}
\subsubsection{NFA - Automa Finito Non-Deterministico}
È una quintupla:
\begin{equation*}
    A=\{Q,\Sigma,\delta,q0,F\}
\end{equation*}
\subparagraph*{$Q$} è un insieme finito di \textbf{stati}.
\subparagraph*{$\Sigma$} è un alfabeto finito (\textbf{simboli in input}).
\subparagraph*{$\delta$} è una funzione di transizione $Q\times\Sigma\rightarrow 2^Q$
\subparagraph*{$q0$} $\in\ Q$ è lo \textbf{stato iniziale}.
\subparagraph*{$F$} è l'insieme degli \textbf{stati finali}.
\subparagraph*{}
Per semplicità possiamo estendere la definizione della funzione di transizione
agli insiemi di stati:
\begin{equation*}
    \delta(\{r_1,r_2,\dots,r_k\},a)=\bigcup\{\delta(r_i,a)\ |\ i=1,2,\dots,k\}
\end{equation*}
La funzione di transizione $\delta$ si può estendere alle stringhe:
\begin{equation*}
    \hat{\delta}:Q\times\Sigma^*\rightarrow 2^Q
\end{equation*}
Definizione:
\begin{equation*}
    \hat{\delta}(q,\epsilon)=\{q\}
\end{equation*}
\begin{equation*}
    \hat{\delta}(q,xa)=\bigcup_{r\in\hat{\delta}(q,x)}\delta(r,a)=\delta(\hat{\delta}(q,x),a)  
\end{equation*}

\subsection{Conversione da NFA a DFA}
\subsubsection{Esempio 1}
NFA:
\begin{center}
    \begin{tabular}{c |c c} 
     & 0 & 1 \\
    \hline
    A & B & $\emptyset$\\
    B & B & B  \\
   \end{tabular}
%    \\
%    \begin{tikzpicture}[shorten >=1pt,node distance=2cm,on grid,auto]
%       \node[state, initial] (A) {$A$};
%       \node[state, accepting, right of=A] (B) {$B$};
%       \draw 
%           % (A) edge[loop above] node{0} (A)
%           (A) edge[above] node{0} (B)
%           (B) edge[loop above] node{0,1} (B);
%   \end{tikzpicture}
\end{center}
DFA:
\begin{center}
    \begin{tabular}{c |c c} 
     & 0 & 1 \\
    \hline
    A & B & C\\
    % \hline
    B & B & B  \\
    C & C & C
   \end{tabular}\\
%    \begin{tikzpicture}[shorten >=1pt,node distance=2cm,on grid,auto]
%       \node[state, initial] (A) {$A$};
%       \node[state, accepting, right of=A] (B) {$B$};
%       \node[state, below of=A] (C) {$C$};
%       \draw 
%           % (A) edge[loop above] node{0} (A)
%           (A) edge[above] node{0} (B);
%           (A) edge[right] node{1} (C)
%           (B) edge[loop above] node{0,1} (B);
%   \end{tikzpicture}
\end{center}
C è lo stato morto, siccome il DFA non accetta $\emptyset$.
\\C viene inserito nella colonna degli stati perché per ogni stato possibile
il DFA richiede un input.
\subsubsection{Esempio 2}
NFA:
\begin{center}
    \begin{tabular}{c |c c} 
     & 0 & 1 \\
    \hline
    A & \{A\} & \{A,B\}\\
    B & $\emptyset$ & $\emptyset$  \\
   \end{tabular}
\end{center}
DFA:
\begin{center}
    \begin{tabular}{c |c c} 
     & 0 & 1 \\
    \hline
    A & \{A\} & \{A,B\}\\
    % \hline
    AB & A$\cup \emptyset=$ \{A\} & \{AB\}
   \end{tabular}\\
\end{center}
AB è uno stato singolo.
\\AB invece di mettere B perché B non può essere raggiunto
da nessuno degli stati precedenti.
\\Nella riga di AB, colonna degli stati, faccio l'unione di A e B per ogni input.
\subsubsection{Esempio 3}
NFA:
\begin{center}
    \begin{tabular}{c |c c} 
     & a & b \\
    \hline
    $\rightarrow$A & \{A,B\} & \{C\}\\
    B & \{A\} & \{B\}  \\
    \tikz\draw[thick](0,0) circle(0.3cm) node{C}; & - & \{A,B\}
   \end{tabular}
\end{center}
DFA:
\begin{center}
    \begin{tabular}{c |c c} 
     & 0 & 1 \\
    \hline
    A & AB & C\\
    AB & AB & CB\\
    \tikz\draw[thick](0,0) circle(0.4cm) node{BC};  & A & AB\\
    \tikz\draw[thick](0,0) circle(0.3cm) node{C}; & D & AB\\
    D & D & D\\
   \end{tabular}\\
\end{center}
BC e C sono stati finali perché sono quelli che contengono la C, che è lo stato
finale dell'NFA.

\section{Grammatiche}
\subsection{Derivazioni di una Grammatica}
La Grammatica è una quadrupla $(V,T,P,S)$:
\begin{itemize}
    \item $V$: insieme delle \textbf{variabili} o \textbf{non terminali}
    \item $T$: insieme dei \textbf{terminali}
    \item $P$: insieme delle \textbf{produzioni} o \textbf{regole di riscrittura}
    \item $S$: il simbolo \textbf{iniziale} o \textbf{assioma}. Da questo si parte, e con le regole, si generano le stringhe che formeranno il linguaggio.
\end{itemize}
Seguendo queste regole si generano delle stringhe che formano un linguaggio.
\begin{itemize}
    \item[Esempio] Consideriamo le seguente grammatica $G_1$
    \begin{equation*}
        G_1=(\{S,A\},\{a,b\},S,\{S \rightarrow aAb,aA \rightarrow aaAb,A\rightarrow \epsilon\})
    \end{equation*}
    \begin{math}
        S \rightarrow \underline{aA}b\\
        \rightarrow a\underline{aA}bb\\
        \rightarrow aaa\underline{A}bbb\\
        \rightarrow aaabbb
    \end{math}
\end{itemize}

\subsection{Grammatiche Context-Free}
La Grammatica Context-Free è una quadrupla $(V,\Sigma,P,S)$:
\begin{itemize}
    \item $V$: insieme delle \textbf{variabili} o \textbf{non terminali}
    \item $\Sigma$: insieme dei \textbf{terminali}
    \item $P$: insieme delle \textbf{produzioni} o \textbf{regole di riscrittura}
    \item $S$: il simbolo \textbf{iniziale} o \textbf{assioma}. Da questo si parte, e con le regole, si generano le stringhe che formeranno il linguaggio.
\end{itemize}
La \textbf{quadrupla} contiene queste regole che, seguite, generano delle stringhe.\\
L'insieme di queste \textbf{stringhe} formano il \textbf{linguaggio} $L$ generato dalla grammatica $G$:
\begin{equation*}
    L(G)
\end{equation*}

\subsection{Alberi Sintattici}
Un albero è un \textbf{albero sintattico} se:
\begin{itemize}
    \item Ogni nodo interno è etichettato con una \textbf{variabile $V$}.
    \item Ogni foglia è etichettata con un simbolo in $V \cup T \cup \{\epsilon\}$.
\end{itemize}
Il \textbf{prodotto} di un albero sintattico è la stringa di foglie da \textbf{sinistra a destra}.\\
Le derivazioni possono essere:
\begin{itemize}
    \item Leftmost - si espandono per prima le \textbf{variabili} più a sinistra.
    \item Rightmost - si espandono per prima le \textbf{variabili} più a destra.
\end{itemize}
Una Grammatica si dice \textbf{ambigua} se esiste \textbf{almeno una} stringa che abbia \textbf{due o più alberi} sintattici.

\subsection{Automi a Pila - Pushdown Automata (PDA)}
Un \textbf{PDA} è un modo per implementare una Context-Free Grammar in un modo simile in cui implementiamo un \textbf{Linguaggio Regolare} usando gli \textbf{Automi a Stati Finiti}.\\
Essi hanno \textbf{più memoria} grazie allo \textbf{stack}.\\
Un PDA ha 3 componenti:
\begin{itemize}
    \item Input - stringa di input
    \item Finite Control Unit - in base all'input fa la \textbf{PUSH} o \textbf{POP} dallo \textbf{stack}
    \item Stack (con memoria infinita)
\end{itemize}
\textbf{PDA} è una tupla di 7 elementi:
\begin{equation*}
    P=(Q,\Sigma,\Gamma,\Delta,q_0,Z_0,F)
\end{equation*}
dove
\begin{itemize}
    \item $Q$
    \item $\Sigma$
    \item $\Gamma$
    \item $\Delta$ è la funzione di transizione
    \item $q_0$
    \item $Z_0$
    \item $F$
\end{itemize}

\subsubsection{$FIRST(X)$ di una variabile in $LL(1)$}
\begin{enumerate}
    \item se $X$ è un simbolo \textbf{terminale} (lettera minuscola) allora $FIRST(X)=X$.
    \item se $X \rightarrow \epsilon$ è una \textbf{produzione} allora \textbf{aggiungi} $FIRST(X)=\{\epsilon\}$.
    \item se $X$ è un \textbf{non-terminale} e $X \rightarrow Y_1,Y_2\dots Y_K$ è una \textbf{produzione} allora \textbf{aggiungi} $FIRST(Y_1)$ al $FIRST(X)$
            se $Y_1 \rightarrow \epsilon$ (deriva $\epsilon$) allora \textbf{aggiungi} $FIRST(Y_2)$ al $FIRST(X)$.
\end{enumerate}

\subsubsection{$FOLLOW(X)$ di una variabile in $LL(1)$}
\begin{enumerate}
    \item se $X$ è il \textbf{simbolo di start} allora $FOLLOW(X)=\$$.
    \item se $A \rightarrow \alpha B \beta$ è una \textbf{produzione} allora $FOLLOW(B)=FIRST(\beta)$ eccetto $\epsilon$, se $B$ contiene $\epsilon$.
    \item se $A \rightarrow \alpha B \beta$ o $A \rightarrow \alpha B \beta$ e $FIRST(\beta)$ contiene $\epsilon$ allora \textbf{aggiungi} $FOLLOW(A)$ al $FOLLOW(B)$.
\end{enumerate}

\newpage

\section{DOMANDE}
\begin{itemize}
    \item Cosa è un linguaggio regolare? Come si verifica?
    \item Differenza tra linguaggi Context-Free e non? Come lo stabilisco?
    \item Quando una grammatica è ambigua?
    \item Quando un automa è minimo?
    \item Quando una grammatica è LL(1)?
\end{itemize}
\end{document}